\abstract[Abstract]{One of the main laws that rule the evolution of software systems states that as the software is continously evolving, its complexity grows and its quality deteriorates unless work is done to mitigate this effect. Traditionally refactoring has been the main type of activity to reduce the complexity growth and quality decay. However, it is still unclear whether this refactoring activies are effective to improve the quality of the resulting code. But there are still open questions regarding what moves developers to refactor certain code entities and the degree of effectivenes of this refactoring activities. In this paper we report the results of an empirical study aiming at analyzing the characteristics of the software entities being refactor and the impact that those refactoring activities have on the maintainability of the system. We have inspected three big software systems, detecting the presence of 26 different types of refactoring operations on thousands of commits. We analyze: i) whether the size, the complexity or the presence of code smells can explain why we choose to refactor a given cod entity and ii) whether the refactoring activities improves the maintainability of the refactored code entity. The results show that XX and YY.   }

\keywords{Class file; \LaTeXe; \emph{Wiley NJD}}